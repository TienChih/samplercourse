\documentclass{ximera}
      
\title{Conclusion}
      
      
\begin{document}
      
\begin{abstract}
      
Some Final Remarks.      
\end{abstract}
      
\maketitle
      
      
These exercises were all written by the author, but as these are standard courses, the exposition or problems likely follow similarly to many texts.  Certainly, extremely good exposition of Calculus I and II using Ximera may be found \href{http://ximera.osu.edu/course/mooculus/calculus1/}{here} and \href{http://ximera.osu.edu/course/mooculus/calculus2/}{here}, by the original founders of Ximera.

The author was first introduced to Ximera through the AMS blog \href{http://blogs.ams.org/phdplus/2016/08/31/the-ximera-project-turning-latex-into-interactive-websites/#sthash.b9OeivZv.dpbs}{``Ph.D.+Epsilon"} written by Dr.\  Sara Malec of Hood College.  By contacting her, I was introduced to Dr.'s Bart Snapp and Jim Fowler of Ohio State University, who are co-PI's of the NSF grant funding this project.  All three have been exceedingly generous with their time and patience, allowing me to become a Ximera author, and answering an exceeding number of technical questions.

\begin{question}
The author is extremely grateful to the following:
\begin{selectAll}

\choice[correct]{Dr.\ Jim Fowler of Ohio State University}

\choice[correct]{Dr.\ Sara Malec of Hood College}
\choice[correct]{Dr.\ Bart Snapp of Ohio State University}

\choice[correct]{The National Science Foundation}
\end{selectAll}
\end{question}

\textbf{\underline{What's Next?}}

In the short term, I am working to create an introductory statistics course in Ximera, using as much as possible open-source free resources such as AIMS approved open textbooks (\url{http://aimath.org/textbooks/approved-textbooks/}) and computation tools (\url{http://www.intro-stats.com/}).  This course will be for my students this coming spring semester, but will be freely available to anyone interested.

In the longer term, I hope to learn more about the in's and out's of Ximera and participate in working on the back-end.  I would like to integrate more free online tools such as Sage Math Cloud and SageCell (\url{https://cloud.sagemath.com/}) a computer algebra system which would be appropriate for advanced undergraduate courses.

If you have any questions or comments, please contact me:\\
\href{mailto:tien.chih@newberry.edu}{tien.chih@newberry.edu}\\
\url{https://sites.google.com/site/drtchih/}



















\end{document}
