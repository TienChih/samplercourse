\documentclass{ximera}
    \usepackage{amsmath, amsthm}
\usepackage{amssymb, amsrefs}
\usepackage{stmaryrd}
\usepackage{graphicx}
\usepackage{setspace}
\usepackage{textcomp}
\usepackage{esvect}
\usepackage{arydshln}
\usepackage{etex}

\usepackage{hyperref}
\usepackage{tikz-cd}
\usetikzlibrary{graphs,graphs.standard}
\usepackage{pgfplots}
\usetikzlibrary{matrix,arrows,decorations.pathmorphing}
\usetikzlibrary{shapes.geometric}

\usepackage[scale=0.8]{geometry} % Reduce document margins


\usepackage[final]{pdfpages}












\newcommand{\bcup}{\displaystyle\bigcup}
\newcommand{\bcap}{\displaystyle\bigcap}
\newcommand{\dsum}{\displaystyle\sum}
\newcommand{\dint}{\displaystyle\int}




\newcommand{\R}[1]{\mathbb{R}^{#1}}
\newcommand{\C}[1]{\mathbb{C}^{#1}}
\newcommand{\Z}[1]{\mathbb{Z}^{#1}}
\newcommand{\embed}[0]{\hookrightarrow}
\newcommand{\Sol}[0]{\textbf{Solution:}\ }

\newcommand{\va}[0]{\mathbf{a}}
\newcommand{\vb}[0]{\mathbf{b}}
\newcommand{\vc}[0]{\mathbf{c}}
\newcommand{\vi}[0]{\mathbf{i}}
\newcommand{\vj}[0]{\mathbf{j}}
\newcommand{\vk}[0]{\mathbf{k}}
\newcommand{\vx}[0]{\mathbf{x}}
\newcommand{\vy}[0]{\mathbf{y}}
\newcommand{\vz}[0]{\mathbf{z}}




\DeclareMathOperator{\Aut}{Aut}
\DeclareMathOperator{\Image}{Im}
\DeclareMathOperator{\Dom}{Dom}
\DeclareMathOperator{\Hom}{Hom}
\DeclareMathOperator{\Ker}{Ker}

   
%      \usepackage{hyperref}
      
\title{Introduction}
      
      
\begin{document}
      
\begin{abstract}
      
Some Introductory Remarks.      
\end{abstract}
      
\maketitle
      
      
This is a sampler course meant to highlight some of the possible uses and capabilities that Ximera could have in certain courses here at Newberry College.  This is by no means an exhaustive list, but is meant to highlight possible applications.

The usual tools available to students in a math class to attain information is the lectures or class time of the instructor, and the explanations and exhibition of the course text.  Reading the text is a good way for students to be given the facts of the course and follow some examples, but reading itself is a passive rather than active process.  As is often said, Mathematics is not a spectator sport, and passive reading will not be sufficient for students to retain information.  This is generally where exercises come into play.  However, if students do not sufficiently understand the material to begin attempting exercises, or if they do the exercises and make some crucial mistake that they don't recognize until the exercises are graded, then this is of little value to the students.

Depending on the teaching style of the instructor, the lecture or class time can be a time for active engagement and interactivity with the material.  But the level of interactivity, and the pacing is a collaboration between all the students in the class and the instructor.  As is so often the case with any collaboration, sometimes the best case scenario is when everyone is only slightly disappointed.

With Ximera, I hope to combine the interactivity of the class room setting with the self-pacing that would come with reading on your own time.  The activities described here will work as a supplement to the traditional learning tools.  Student's can work through problems where the strategies are laid out and they are able to follow these examples while being asked to fill in crucial gaps in the problem.  This would then be followed by exercises with less hand holding, until students are able to independently devise strategies to problems and find the solutions.

Ximera also allows integration with Desmos (\url{desmos.com}), a fantastic dynamic computation and graphing tool.  By using Desmos graphs, students will be able to quickly and easily modify the graphs in the exposition or the exercises to create their own examples, and see the concepts of the course come alive in front of them.  This grants students a stronger intuitive grasp of the core ideas of the course.

Lastly, we live  in an age of unprecedented information democratization.  Those of us who are in academia, and thus the business of knowledge, must be an active part of this democratization.  Proprietary online homework systems, while certainly having high utility,  are often a heavy financial burden on lower income students.  These systems come bundled with textbooks, but because of the online access, no secondary or used textbook market exists for those with less financial means.  This trend is counter to the spirit of open access to knowledge that defines this age.  Ximera would allow for instructors to create and customize their own online content and exercises that could fit their own lectures and material of their choosing.  Since no particular texts are bundled with it, instructors would be free to use less expensive, or free open-source alternatives, relieving the burden for many students.


If you have any questions or comments, please contact me:\\
\href{mailto:tien.chih@newberry.edu}{tien.chih@newberry.edu}\\
\url{https://sites.google.com/site/drtchih/}














\end{document}
